\documentclass[12pt,a4paper]{article}
\usepackage[utf8]{inputenc}
\usepackage[spanish]{babel}
\usepackage{amsmath}
\usepackage{amsfonts}
\usepackage{amssymb}
\usepackage{graphicx}
\usepackage{float}
%\usepackage[left=2cm,right=2cm,top=2cm,bottom=2cm]{geometry}
\usepackage{geometry}
\geometry{ a4paper, total={185mm,265mm}}
\usepackage{multicol}
\usepackage{lscape}
\usepackage{physics}
\usepackage{subfig}
\usepackage{hyperref}

\usepackage{wrapfig}
\usepackage{xcolor}
\usepackage{listings}

\hypersetup{
   colorlinks=true,
   urlcolor=blue,
   linkcolor=blue,
   citecolor=blue,
   %filecolor=magenta,
}

\definecolor{mygray}{RGB}{250, 249, 227}
\definecolor{codegreen}{rgb}{0,0.6,0}
\lstdefinestyle{mystyle}{
  backgroundcolor=\color{mygray},   commentstyle=\color{codegreen},
  keywordstyle=\color{blue},
  numberstyle=\tiny\color{gray},
  stringstyle=\color{purple},
  basicstyle=\ttfamily\footnotesize,
  breakatwhitespace=false,         
  breaklines=true,                 
  captionpos=b,                    
  keepspaces=true,                 
  numbers=left,                    
  numbersep=5pt,                  
  showspaces=false,                
  showstringspaces=false,
  showtabs=false,                  
  tabsize=2
}
%"mystyle" code listing set
\lstset{style=mystyle}


\title{PL2: Análisis de datos y regresión}
\author{Aitor Lorenzo Ramírez Cabrera}
\date{Octubre 2021}

\begin{document}

\begin{center}
    \textbf{MODELIZACIÓN ESTADÍSTICA 2021-2022\\ Máster en modelización e investigación matemática, estadística y computación\\[3mm] Evaluación continua 2}\\[1mm] \textsc{Aitor Lorenzo Ramírez Cabrera} \\[1mm]
\end{center}

\section{Teoría: Define los siguientes conceptos estadísticos y explica su significado}

\begin{itemize}
    \item \textbf{Puntuación tipificada:} Se utiliza para comparar las posiciones relativas de varios elementos con respecto al conjunto de observaciones. Una puntuación tipificada se puede calcular como:
    \begin{equation}
        z_i = \frac{x_i - \Bar{X}}{s}
    \end{equation}
    donde $\Bar{X}$ es la media de las observaciones y $s$ la desviación estándar. Como podemos observar, $z_i$ no es más que el número de desviaciones estándar que $x_i$ se desvía de la media. La puntuación tipificada es útil para comparar datos procedente de diferentes muestras.
    %si tal añadir algo de las propiedades
    
    \item \textbf{Coeficiente de correlación de Pearson:} Es una prueba que mide la dependencia lineal entre dos variables cuantitativas continuas.\\ Este puede tomar valores en un rango $[-1,1]$, siendo 0 la prueba de que no hay asociación entre las variables. Un valor mayor que cero indica una asociación positiva, esto es, a medida que aumenta una variable también lo hace la otra. Por otra parte, un valor menor que cero indica una asociación negativa, esto es, a medida que aumenta una variable la otra disminuye.\\
    Para una población, dado un par de variables aleatorias $(X,Y)$ se define como:
    \begin{equation}
        \rho_{X,Y} = \frac{Cov(X,Y)}{\sqrt{Var(X) Var(Y)}}
    \end{equation}
    Mientras que, para una muestra dada por n pares de datos $\{(x_i, y_i)\}^n_{i=1}$ se define como:
\begin{equation}
    r_{xy} = \frac{\sum_{i=1}^n(x_i-\Bar{x})(y_i-\Bar{y})}{\sqrt{\sum_{i=1}^n(x_i-\Bar{x})^2}\sqrt{\sum_{i=1}^n(y_i-\Bar{y})^2}}
\end{equation}
    
    \item \textbf{Intervalo de confianza:} Es un par de números entre los cuales se encontrará la estimación puntual buscada. El intervalo de confianza nos permite calcular dos valores alrededor de una media muestral que acoten un rango dentro del cual se va a localizar el parámetro poblacional.
    
    \item \textbf{Región crítica de un test:} La región crítica a un nivel de significación $\alpha$ representa el subconjunto del espacio muestral tal que la probabilidad de que la muestra aleatoria simple pertenezca a esta. Cuando se cumple la hipótesis nula, $H_0$, esta es igual a $\alpha$, es decir:
    \begin{equation}
        Pr((\xi_1, \xi_2, ..., \xi_n)\in C|H_0) = \alpha
    \end{equation}
    La regla de decisión en un test quedará definida de acuerdo a una región crítica. Si la muestra obtenida se ubica dentro de la región crítica, rechazamos la hipótesis nula $H_0$. En caso contrario, no rechazamos la hipótesis nula.
    
    \item \textbf{p-valor del contraste:} Es la probabilidad de obtener un valor del estadístico al menos tan extremo como el que se ha observado si la hipótesis nula es cierta. Se puede decir que representa la probabilidad de observar la muestra cuando la hipótesis nula es cierta.\\
    Si el p-valor es muy pequeño ($p < 0.05$) la muestra es poco compatible con que $H_0$ sea cierta y se rechaza $H_0$.\\
    Si el p-valor no es pequeño ($p \geq 0.05$), la muestra es compatible con que $H_0$ sea cierta y no se rechaza.
    
    \item \textbf{Regresión logística:} Es un método de regresión que permite estimar la probabilidad de una variable cualitativa binaria en función de una variable cuantitativa.
    
    
\end{itemize}

\section{Simulación}

\end{document}
