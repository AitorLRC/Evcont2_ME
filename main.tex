\documentclass[12pt,a4paper]{article}
\usepackage[utf8]{inputenc}
\usepackage[spanish]{babel}
\usepackage{amsmath}
\usepackage{amsfonts}
\usepackage{amssymb}
\usepackage{graphicx}
\usepackage{float}
%\usepackage[left=2cm,right=2cm,top=2cm,bottom=2cm]{geometry}
\usepackage{geometry}
\geometry{ a4paper, total={185mm,265mm}}
\usepackage{multicol}
\usepackage{lscape}
\usepackage{physics}
\usepackage{subfig}
\usepackage{hyperref}

\usepackage{wrapfig}
\usepackage{xcolor}
\usepackage{listings}

\hypersetup{
   colorlinks=true,
   urlcolor=blue,
   linkcolor=blue,
   citecolor=blue,
   %filecolor=magenta,
}

\definecolor{mygray}{RGB}{250, 249, 227}
\definecolor{codegreen}{rgb}{0,0.6,0}
\lstdefinestyle{mystyle}{
  backgroundcolor=\color{mygray},   commentstyle=\color{codegreen},
  keywordstyle=\color{blue},
  numberstyle=\tiny\color{gray},
  stringstyle=\color{purple},
  basicstyle=\ttfamily\footnotesize,
  breakatwhitespace=false,         
  breaklines=true,                 
  captionpos=b,                    
  keepspaces=true,                 
  numbers=left,                    
  numbersep=5pt,                  
  showspaces=false,                
  showstringspaces=false,
  showtabs=false,                  
  tabsize=2,
}
%"mystyle" code listing set
\lstset{style=mystyle,
    inputencoding=utf8,
    literate      =        % Support additional characters
      {á}{{\'a}}1  {é}{{\'e}}1  {í}{{\'i}}1 {ó}{{\'o}}1  {ú}{{\'u}}1
      {Á}{{\'A}}1  {É}{{\'E}}1  {Í}{{\'I}}1 {Ó}{{\'O}}1  {Ú}{{\'U}}1
      {à}{{\`a}}1  {è}{{\`e}}1  {ì}{{\`i}}1 {ò}{{\`o}}1  {ù}{{\`u}}1
      {À}{{\`A}}1  {È}{{\'E}}1  {Ì}{{\`I}}1 {Ò}{{\`O}}1  {Ù}{{\`U}}1
      {ä}{{\"a}}1  {ë}{{\"e}}1  {ï}{{\"i}}1 {ö}{{\"o}}1  {ü}{{\"u}}1
      {Ä}{{\"A}}1  {Ë}{{\"E}}1  {Ï}{{\"I}}1 {Ö}{{\"O}}1  {Ü}{{\"U}}1
      {â}{{\^a}}1  {ê}{{\^e}}1  {î}{{\^i}}1 {ô}{{\^o}}1  {û}{{\^u}}1
      {Â}{{\^A}}1  {Ê}{{\^E}}1  {Î}{{\^I}}1 {Ô}{{\^O}}1  {Û}{{\^U}}1
      {œ}{{\oe}}1  {Œ}{{\OE}}1  {æ}{{\ae}}1 {Æ}{{\AE}}1  {ß}{{\ss}}1
      {ç}{{\c c}}1 {Ç}{{\c C}}1 {ø}{{\o}}1  {Ø}{{\O}}1   {å}{{\r a}}1
      {Å}{{\r A}}1 {ã}{{\~a}}1  {õ}{{\~o}}1 {Ã}{{\~A}}1  {Õ}{{\~O}}1
      {ñ}{{\~n}}1  {Ñ}{{\~N}}1  {¿}{{?`}}1  {¡}{{!`}}1
      {°}{{\textdegree}}1 {º}{{\textordmasculine}}1 {ª}{{\textordfeminine}}1
    }


\title{PL2: Análisis de datos y regresión}
\author{Aitor Lorenzo Ramírez Cabrera}
\date{Octubre 2021}

\begin{document}

\begin{center}
    \textbf{MODELIZACIÓN ESTADÍSTICA 2021-2022\\ Máster en modelización e investigación matemática, estadística y computación\\[3mm] Evaluación continua 2}\\[1mm] \textsc{Aitor Lorenzo Ramírez Cabrera} \\[1mm]
\end{center}

\section{Teoría: Define los siguientes conceptos estadísticos y explica su significado}

\begin{itemize}
    \item \textbf{Puntuación tipificada:} Se utiliza para comparar las posiciones relativas de varios elementos con respecto al conjunto de observaciones. Una puntuación tipificada se puede calcular como:
    \begin{equation}
        z_i = \frac{x_i - \Bar{X}}{s}
    \end{equation}
    donde $\Bar{X}$ es la media de las observaciones y $s$ la desviación estándar. Como podemos observar, $z_i$ no es más que el número de desviaciones estándar que $x_i$ se desvía de la media. La puntuación tipificada es útil para comparar datos procedente de diferentes muestras.
    %si tal añadir algo de las propiedades
    
    \item \textbf{Coeficiente de correlación de Pearson:} Es una prueba que mide la dependencia lineal entre dos variables cuantitativas continuas.\\ Este puede tomar valores en un rango $[-1,1]$, siendo 0 la prueba de que no hay asociación entre las variables. Un valor mayor que cero indica una asociación positiva, esto es, a medida que aumenta una variable también lo hace la otra. Por otra parte, un valor menor que cero indica una asociación negativa, esto es, a medida que aumenta una variable la otra disminuye.\\
    Para una población, dado un par de variables aleatorias $(X,Y)$ se define como:
    \begin{equation}
        \rho_{X,Y} = \frac{Cov(X,Y)}{\sqrt{Var(X) Var(Y)}}
    \end{equation}
    Mientras que, para una muestra dada por n pares de datos $\{(x_i, y_i)\}^n_{i=1}$ se define como:
\begin{equation}
    r_{xy} = \frac{\sum_{i=1}^n(x_i-\Bar{x})(y_i-\Bar{y})}{\sqrt{\sum_{i=1}^n(x_i-\Bar{x})^2}\sqrt{\sum_{i=1}^n(y_i-\Bar{y})^2}}
\end{equation}
    
    \item \textbf{Intervalo de confianza:} Es un par de números entre los cuales se encontrará la estimación puntual buscada. El intervalo de confianza nos permite calcular dos valores alrededor de una media muestral que acoten un rango dentro del cual se va a localizar el parámetro poblacional.
    
    \item \textbf{Región crítica de un test:} La región crítica a un nivel de significación $\alpha$ representa el subconjunto del espacio muestral tal que la probabilidad de que la muestra aleatoria simple pertenezca a esta. Cuando se cumple la hipótesis nula, $H_0$, esta es igual a $\alpha$, es decir:
    \begin{equation}
        Pr((\xi_1, \xi_2, ..., \xi_n)\in C|H_0) = \alpha
    \end{equation}
    La regla de decisión en un test quedará definida de acuerdo a una región crítica. Si la muestra obtenida se ubica dentro de la región crítica, rechazamos la hipótesis nula $H_0$. En caso contrario, no rechazamos la hipótesis nula.
    
    \item \textbf{p-valor del contraste:} Es la probabilidad de obtener un valor del estadístico al menos tan extremo como el que se ha observado si la hipótesis nula es cierta. Se puede decir que representa la probabilidad de observar la muestra cuando la hipótesis nula es cierta.\\
    Si el p-valor es muy pequeño ($p < 0.05$) la muestra es poco compatible con que $H_0$ sea cierta y se rechaza $H_0$.\\
    Si el p-valor no es pequeño ($p \geq 0.05$), la muestra es compatible con que $H_0$ sea cierta y no se rechaza.
    
    \item \textbf{Regresión logística:} Es un método de regresión que permite estimar la probabilidad de una variable cualitativa binaria en función de una variable cuantitativa.
    
    
\end{itemize}

\section{Simulación}
\subsection{Analiza mediante simulación el efecto que producen los modelos de regresión diversos factores.}
Para empezar, generamos las variables independientes de la simulación tal y como se muestra a continuación:
\begin{lstlisting}[language=R, caption = Generación de variables independientes, label =cod:1]
n <- 40 #Tamaño de muestra al menos de 3 observaciones

set.seed(1); x1 <- rnorm(n) #Genero la variable aleatoria x1
#Para generar la segunda variable con correlación de poisson >0.1 creo la función
corr.data<- function(x1, rho){
  set.seed(7);xr <- rnorm(length(x1)) #Variable random con distribución normal
  
  xcorr<- rho*x1 + sqrt(1-rho^2)*xr #Genero la variable correlada
  
  return(xcorr)
}
x2 <- corr.data(x1, 0.2) #Genero x2 con correlación 0.2 con respecto a x1
set.seed(2); x3 <- rnorm(n) #Genero la variable aleatoria x3

\end{lstlisting}
A continuación, para calcular la variable dependiente defino unos valores cualesquiera para los $\beta$, unos residuales aleatorios de media nula y desviación típica $\sigma$ y aplico la fórmula de regresión lineal:
\begin{equation}
    y = \beta_0 + \beta_1 x_1 + \beta_2 x_2 + \beta_3 x_3 + res
\end{equation}
para el caso de estudio, los valores de $\beta$ elegidos son:
\begin{equation*}
    \beta_0 = 10 \ \ \ ;\ \ \ \beta_1 = 5 \ \ \ ;\ \ \ \beta_2 = 23 \ \ \ ;\ \ \  \beta_3 = 15
\end{equation*}
Ahora, teniendo la variable dependiente y las independientes, pasamos a crear el modelo de regresión lineal con la función \textit{lm()} de R de manera que obtenemos:
\begin{lstlisting}[language=R, caption = Resultado de la regresión lineal, label =cod:2]
> summary(reg1)

Call:
lm(formula = y ~ x1 + x2 + x3)

Residuals:
    Min      1Q  Median      3Q     Max 
-9.3047 -3.2820 -0.5669  2.6613  9.3872 

Coefficients:
            Estimate Std. Error t value Pr(>|t|)    
(Intercept)   9.9420     0.7825  12.706 7.20e-15 ***
x1            5.2190     0.8972   5.817 1.22e-06 ***
x2           22.8265     0.7461  30.596  < 2e-16 ***
x3           14.3133     0.6934  20.641  < 2e-16 ***
---
Signif. codes:  0 '***' 0.001 '**' 0.01 '*' 0.05 '.' 0.1 ' ' 1

Residual standard error: 4.755 on 36 degrees of freedom
Multiple R-squared:  0.978,	Adjusted R-squared:  0.9762 
F-statistic: 533.7 on 3 and 36 DF,  p-value: < 2.2e-16
\end{lstlisting}
En \hyperref[cod:2]{Listing \ref{cod:2}} podemos observar que, como era de esperar, los valores de $\beta$ son similares a los que habíamos definido anteriormente. Además, en todos los casos el p-valor es mucho menor que $0.05$ por lo tanto rechazamos la hipótesis nula, es decir, rechazamos que alguno de estos $\beta$ sea nulo.

Procedemos a analizar el efecto que producen en los modelos de regresión los siguientes factores:
\subsubsection{Un punto de influencia}
Para comprobar esto, sabemos que un punto de influencia es un valor atípico que afecta a la pendiente de la línea de regresión. Por lo tanto, podemos alterar alguno de los valores del cualquier variable independiente de manera que sea mucho mayor que el resto.

Empezaremos comprobando que pasa si el punto se altera en la variable que tiene el menor $\beta$. En el caso de estudio se tomó la quinta componente de $x_1$ y se le sumó quince unidades para obtener un valor mucho mayor que el resto. Los resultados de la regresión obtenida en este caso fueron:
\begin{lstlisting}[language=R, caption = Regresión lineal con un punto de influencia, label =cod:3]
> summary(reg2)

Call:
lm(formula = y ~ x1.1 + x2 + x3)

Residuals:
    Min      1Q  Median      3Q     Max 
-8.9560 -4.9148  0.0496  3.1637 14.5616 

Coefficients:
            Estimate Std. Error t value Pr(>|t|)    
(Intercept)   9.6070     1.0415   9.224 5.14e-11 ***
x1.1          0.8618     0.3888   2.216   0.0331 *  
x2           24.0877     0.9464  25.452  < 2e-16 ***
x3           14.8765     0.8957  16.609  < 2e-16 ***
---
Signif. codes:  0 '***' 0.001 '**' 0.01 '*' 0.05 '.' 0.1 ' ' 1

Residual standard error: 6.213 on 36 degrees of freedom
Multiple R-squared:  0.9625,	Adjusted R-squared:  0.9593 
F-statistic: 307.7 on 3 and 36 DF,  p-value: < 2.2e-16
\end{lstlisting}
Comparando los resultados obtenidos en \hyperref[cod:3]{Listing \ref{cod:3}} con los de \hyperref[cod:2]{Listing \ref{cod:2}} observamos que apenas hay diferencia en los valores de $\beta_0$, $\beta_2$ y $\beta_3$ (asociados a las variables que no han sido modificadas). Pero sí que hay gran diferencia entre las $\beta_1$ ya que vemos que esta pasa de ser $5.22$ a ser $0.86$. Además, se observa diferencia en el p-valor de esta última, que es mucho mayor, aunque seguimos rechazando la hipótesis de que $\beta_1$ pueda ser nula.

A modo de curiosidad podemos comprobar que pasaría si el punto se altera en la variable que tiene el $\beta$ mayor. En el caso de estudio se tomó la quinta componente de $\beta_2$ y se le sumó veinte unidades. Los resultados de la regresión obtenida en este caso fueron:
\begin{lstlisting}[language=R, caption = Regresión lineal con un punto de influencia, label =cod:4]
> summary(reg3)

Call:
lm(formula = y ~ x1 + x2.1 + x3)

Residuals:
    Min      1Q  Median      3Q     Max 
-45.536 -15.191  -4.421  12.978  62.121 

Coefficients:
            Estimate Std. Error t value Pr(>|t|)    
(Intercept)   15.069      4.014   3.754 0.000613 ***
x1            11.629      4.507   2.580 0.014107 *  
x2.1           1.062      1.256   0.846 0.403396    
x3            13.571      3.569   3.802 0.000534 ***
---
Signif. codes:  0 '***' 0.001 '**' 0.01 '*' 0.05 '.' 0.1 ' ' 1

Residual standard error: 24.47 on 36 degrees of freedom
Multiple R-squared:  0.4178,	Adjusted R-squared:  0.3693 
F-statistic: 8.612 on 3 and 36 DF,  p-value: 0.0001931
\end{lstlisting}
En este caso se observa que el efecto es mucho mayor ya que, para empezar, existe más error en el cálculo de los coeficientes de regresión $\beta$. Observando los p-valores, se observa además que estos son mucho mayores que en \hyperref[cod:2]{Listing \ref{cod:2}}, llegando incluso a no rechazar la hipótesis nula para $\beta_2$, es decir, esta variable no sería significativa en el estudio. Se observa, además, que la desviación estándar de los predictores es mucho mayor que la obtenida en el modelo inicial. Por último, llama la atención el $R^2$, que es mucho menor que el de la regresión original, por lo tanto, en estudios de regresión lineal en los que se nos proporcionan los datos podríamos detectar puntos de influencia en función del $R^2$.

\subsubsection{La multicolinealidad}
La multicolinealidad es la alta correlación entre dos variables explicativas de una regresión. Para simular una regresión en la que existe multicolinealidad podemos definir que una de las variables sea igual a otra más unos residuos random de manera que obtengamos una correlación alta. En el caso de estudio se definió:
\begin{equation*}
    x_{2.2} = 2x_1 + N(0,1)
\end{equation*}
Los resultados obtenidos para la regresión en este caso fueron:
\begin{lstlisting}[language=R, caption = Regresión lineal con multicolinealidad, label =cod:5]
> summary(reg4)

Call:
lm(formula = y ~ x1 + x2.2 + x3)

Residuals:
    Min      1Q  Median      3Q     Max 
-41.107 -16.572  -5.198  12.772  64.116 

Coefficients:
            Estimate Std. Error t value Pr(>|t|)    
(Intercept)  16.0409     4.0206   3.990 0.000311 ***
x1           13.7511     8.9001   1.545 0.131081    
x2.2         -0.9659     4.5815  -0.211 0.834215    
x3           13.5471     3.6409   3.721 0.000675 ***
---
Signif. codes:  0 '***' 0.001 '**' 0.01 '*' 0.05 '.' 0.1 ' ' 1

Residual standard error: 24.7 on 36 degrees of freedom
Multiple R-squared:  0.407,	Adjusted R-squared:  0.3576 
F-statistic: 8.235 on 3 and 36 DF,  p-value: 0.000266
\end{lstlisting}
En este caso se observa que tanto el coeficiente independiente de correlación como los asociados a las dos variables correladas varían bastante con respecta a las definidas inicialmente. Con respecto a los p-valores, estos son mucho mayores que los obtenidos en \hyperref[cod:2]{Listing \ref{cod:2}}, llegando incluso a no rechazar la hipótesis de que $\beta_1$ o $\beta_2$ puedan ser nulas. Al igual que en el caso anterior, se observa un valor muy bajo para $R^2$ que, junto a los altos errores estándar y a los p-valores, nos podrían indicar la existencia de multicolinealidad.

\subsubsection{Una especificación inadecuada del modelo}
Una especificación inadecuada del modelo puede venir dada por la omisión de regresores relevantes en el modelo. Teniendo en cuenta lo obtenido en \hyperref[cod:2]{Listing \ref{cod:2}} podemos plantear que pasaría si quitamos una de las variables con un p-valor menor, por ejemplo $x_3$:

\begin{lstlisting}[language=R, caption = Especificación inadecuada del modelo, label =cod:6]
> summary(reg5)  

Call:
lm(formula = y ~ x1 + x2)

Residuals:
    Min      1Q  Median      3Q     Max 
-43.287  -8.914  -1.812  11.128  33.322 

Coefficients:
            Estimate Std. Error t value Pr(>|t|)    
(Intercept)   11.261      2.756   4.086 0.000226 ***
x1             8.047      3.133   2.568 0.014388 *  
x2            22.185      2.634   8.422 3.97e-10 ***
---
Signif. codes:  0 '***' 0.001 '**' 0.01 '*' 0.05 '.' 0.1 ' ' 1

Residual standard error: 16.8 on 37 degrees of freedom
Multiple R-squared:  0.7178,	Adjusted R-squared:  0.7025 
F-statistic: 47.05 on 2 and 37 DF,  p-value: 6.852e-11
\end{lstlisting}
En este caso, la estimación de los coeficientes de regresión es similar a la propuesta inicialmente pero se observa que el error estándar es mayor que el obtenido en \hyperref[cod:2]{Listing \ref{cod:2}}. Comparando los p-valores, en todos los casos se rechaza la hipótesis nula, es decir, todas las variables son significativas, pero estos son mayores que los obtenidos en el modelo inicial.

A continuación comprobamos que ocurre si la especificación inadecuada se produce por la falta de la variable que definimos con un mayor coeficiente de regresión, es decir, si quitamos del análisis $x_2$.

\begin{lstlisting}[language=R, caption = Especificación inadecuada del modelo, label =cod:7]
> summary(reg7)

Call:
lm(formula = y ~ x1 + x3)

Residuals:
   Min     1Q Median     3Q    Max 
-40.85 -16.45  -5.28  12.88  64.17 

Coefficients:
            Estimate Std. Error t value Pr(>|t|)    
(Intercept)   15.869      3.886   4.084 0.000228 ***
x1            12.133      4.450   2.726 0.009728 ** 
x3            13.430      3.551   3.782 0.000551 ***
---
Signif. codes:  0 '***' 0.001 '**' 0.01 '*' 0.05 '.' 0.1 ' ' 1

Residual standard error: 24.37 on 37 degrees of freedom
Multiple R-squared:  0.4062,	Adjusted R-squared:  0.3741 
F-statistic: 12.66 on 2 and 37 DF,  p-value: 6.483e-05
\end{lstlisting}
En este caso se observa que el $R^2$ es mucho mejor que en los casos anteriores, demostrando así la especificación inadecuada del modelo.

\subsection{Simula un conjunto de datos a partir de un modelo de regresión logística}
Para empezar, conocemos que la probabilidad para una regresión logística tiene la forma:
\begin{equation}
    p_i = \frac{1}{1+e^{-(\beta_0 + \beta_1 x_{1,i}+\beta_{2s} x_{2s,i}+\beta_{2n}x_{2n,i})}}
\end{equation}
de manera que, para acotar el programa, nos interesa tomar un rango de valores pequeño que asegure que $p_i$ abarque todo el intervalo $[0,1]$. Si definimos:
\begin{equation}
    K_i = \beta_0 + \beta_1 x_{1,i}+\beta_{2s} x_{2s,i}+\beta_{2n}x_{2n,i}
\end{equation}
Obtenemos que, para que la probabilidad abarque todo ese intervalo, los $K_i$ se deben encontrar en el intervalo $[-5,5]$.



A continuación, podemos definir una variable cualitativa que tome, de manera aleatoria, el valor \textit{s} o el valor \textit{n}. Y, para definir la variable cuantitativa, podemos despejar de la expresión de $K_i$ de manera que obtenemos:
\begin{equation}
    x_{1,i} = \frac{K_i - \beta_{2s} x_{2s,i}-\beta_{2n}x_{2n,i}}{\beta_1}
\end{equation}
En el caso de estudio, los valores de $\beta$ elegidos son:
\begin{equation*}
    \beta_0 = 5 \ \ \ ;\ \ \ \beta_{1s} = 3 \ \ \ ;\ \ \ \beta_{1n} = 2 \ \ \ ;\ \ \  \beta_{2} = 3
\end{equation*}

A continuación, se generan unos nuevos valores $K'$ a partir de estas variables y se le suma una componente aleatoria con una distribución $N(0,0\text{.}8)$. 
Finalmente, para obtener la variable respuesta, se calculan nuevas probabilidades aleatorias a partir de este nuevo $K'$ y se simula la asignación de $Y=1$ según una distribución binomial dependiente de $x$. En R esta simulación se consigue haciendo uso de la función \textit{rbinom()}.

Teniendo el conjunto de datos simulados, podemos hacer una regresión logística en R de manera que obtenemos:
\begin{lstlisting}[language=R, caption = Regresión logística, label =cod:8]
> modelo <- glm(y~x1+x2, family = binomial)
> summary(modelo)

Call:
glm(formula = y ~ x1 + x2, family = binomial)

Deviance Residuals: 
    Min       1Q   Median       3Q      Max  
-1.3836  -0.8201  -0.3268   0.5984   2.0879  

Coefficients:
            Estimate Std. Error z value Pr(>|z|)  
(Intercept)   4.8485     2.0162   2.405   0.0162 *
x1s           0.4319     1.0133   0.426   0.6699  
x2            2.0268     0.7885   2.570   0.0102 *
---
Signif. codes:  0 '***' 0.001 '**' 0.01 '*' 0.05 '.' 0.1 ' ' 1

(Dispersion parameter for binomial family taken to be 1)

    Null deviance: 40.381  on 29  degrees of freedom
Residual deviance: 28.835  on 27  degrees of freedom
AIC: 34.835

Number of Fisher Scoring iterations: 5
\end{lstlisting}
En este caso observamos que, salvo para el caso en el que $x_1 = \textit{s}$ en el que es mucho menor que el indicado. Además, para este caso, se obtiene un p-valor de $0.67$, es decir, este coeficiente podría ser nulo $67$ de cada $100$ veces.

El código usado para resolver los primeros apartados se muestra en el siguiente \textit{link}: \url{https://github.com/AitorLRC/Evcont2_ME/blob/2aed07162522b074ef2356dd28a0fdc7a0061ad2/continua2.R}
\end{document}